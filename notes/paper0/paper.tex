\documentclass{article}
\usepackage{graphicx}


% The purpose of this paper is to convey my idea
% My idea is that gradual typing can be efficiently implemented


% The papers that follow may present some notion of how I can
% present this idea
%% 
%%
%% First-class Runtime Generation of High-performance Types using Exotypes 
%% Zachary DeVito, Daniel Ritchie, Matt Fisher, Alex Aiken, Pat Hanrahan (Stanford University)
%% Getting F-Bounded Polymorphism into Shape 
%% Ben Greenman, Fabian Muehlboeck, Ross Tate (Cornell University)
%% Optimal Inference of Fields in Row-Polymorphic Records 
%% Axel Simon (Technische Universität München)

                                   
\begin{document}
\section{Abstract} % 4 sentances

% I need something novel in order to catch people's attention
% this is used by the programming committee to decide which papers to read
% Write this last

% The problem

% Why this is an interesting problem

% What my solution achieves

% What this implies, Where this leads

\section{Introduction} % 1 page
% The introduction including the contributions should survey the whole paper 

% Describe the problem


% State my contributions
% This is where I should start
% This list will drive the entire paper
% Use forward references to layout the paper

\begin{itemize}
\item We provide implementation strategies and analysis for various mechinisms of
  a gradual type system.
  \begin{itemize}
  \item The includes space efficient casts with regards
    to tail calls and function casts.
  \item Support for both Guarded and Monotonic semantics for references.
  \item First class polymorphism
  \end{itemize}
  \item We provide evaluation of a compiler based implementation
  of the language.
  \begin{itemize}
    \item dynamic invokation (invoking closures versus dynamics values)
    \item cast based versus coersion based intermediate languages
    \item boxed versus unbox integers
    \item space efficiency versus non-space-efficient perfomance and
    and analysis of the time complexity
  \end{itemize}
\item We identify several areas for optimization and the results of those optimizations.
  \begin{itemize}
    \item common subexpression elimination on type lead to faster dynamics
    \item type inference type to eliminate uneaded casts 
    \item cast driven accounting for inline and code specialization
  \end{itemize}
\item As an artifact of this study we have create a compiler for the schml programming
  language. This compiler comes with the test suite used to decide implementation
  strategies at various stages of implementations.
\end{itemize}


% The body of the paper
% This section should present

\section{The Problem}  % 1 page

% present the problem using running examples

Most research regarding gradual types have come up with some logical
foundation for the way that the type system should behave. Then implemented
there system on top a Dynamic type system. This seems to miss the mark on
one of the incentives for using static types -- performance improvements.
Sometimes theses implementations are able to use type information to elide checks
or to provably use unsafe varients of accessors, but the question remains how
much performance is lost to various overheads of using dynamic types.


\section{My Idea} % 2 pages

\section{The Details} % 5 pages


\section{Related Work}  % 1-2 pages
% Give everyone credit and more

\section{Conclusion & Further Work} % 1/2 page

\end{document}
