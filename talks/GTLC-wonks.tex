\documentclass{beamer}

% \usepackage{beamerthemesplit} // Activate for custom appearance
\usepackage{amsmath}
\usepackage{mathtools}
\usepackage{ amssymb }

\title{Compilation of the Gradually-Typed Lambda Calculus}
\author{Andre Kuhlenschmidt}
\date{\today}

\newcommand{\If}[3]{\mathit{if} \, #1 \, \mathit{then} \, #2 \, \mathit{else} \, #3}

\begin{document}

\frame{\titlepage}

\frame
{
  \frametitle{What is the GTLC?}
}

\frame
{
  	\frametitle{What is the GTLC?}
	\begin{figure}[h]
	\centering
	\[
	\begin{array}{lrcl}
		\text{Terms:} & t & ::= & x \mid \lambda x{:}T.t \mid t \, t \\
					  %&   &     & \If{t}{t}{t} \\
		\text{Types:} & T & ::= & T \rightarrow T \\
		\text{Value:} & v & ::= & \lambda x{:}T.t \\%\mid true \mid false \\
		\\
		\text{Typing:} & \frac{x:T \in \Gamma}{\Gamma \vdash x : T}
					   & \frac{\Gamma,x:T_1 \vdash e : T_2}
					         {\Gamma \vdash \lambda x:T_1.e : T_1 \to T_2} 
					   & \frac{\Gamma \vdash t_1 : T_1 \rightarrow T_2 \quad 
					           \Gamma \vdash t_2 : T_1}
					          {\Gamma \vdash t_1 \, t_2 : T_2}\\
		\\	
		\text{Eval:} & \frac{t_1 \rightarrow t_{1}^{\prime}}
					        {t_1 \, t_2 \rightarrow t_{2}^{\prime} \, t_2}
					 & \frac{t_2 \rightarrow t_{2}^{\prime}}
						    {v_1 \, t_2 \rightarrow v_1 \, t_{2}^{\prime}}
		             & \left( \lambda x {:} T_1 . t_12 \right) \, v_2 	  
		               \rightarrow \left[ x \mapsto v2 \right] t_12
		\\
	\end{array}
	\]
	\caption{TAPL definition of the STLC}
	\end{figure}
}

\frame
{
  	\frametitle{What is the GTLC?}
	\begin{figure}[h]
	\centering
	\[
	\begin{array}{lrcl}
		\text{Terms:} & t & ::= & x \mid \lambda x{:}T.t \mid t \, t \\
					  %&   &     & \If{t}{t}{t} \\
		\text{Types:} & T & ::= & ? \mid T \rightarrow T \\			 
	\end{array}
	\]
	\caption{STLC Types extended with the Dynamic Type}
	\end{figure}
}

\frame
{
  	\frametitle{What is the GTLC?}
	\begin{figure}[h]
	\centering
	\[
	\begin{array}{lrcl}
		\text{Terms:} & t & ::= & x \mid \lambda x{:}T.t \mid t \, t \\
					  %&   &     & \If{t}{t}{t} \\
		\text{Types:} & T & ::= & ? \mid T \rightarrow T \\			 
		\text{Uncasted Values:} & u & ::= & \lambda x{:}T.t \\
		\text{Values:} & v & ::= & u \mid \left(u, T \right) \mid \, error \\
	\end{array}
	\]
	\caption{STLC Types extended with the Dynamic Type}
	\end{figure}
}

\frame
{
  	\frametitle{What is the GTLC?}
	\begin{figure}[h]
	\centering
	\[
	\begin{array}{lrcl}
		\text{Terms:} & t & ::= & x \mid \lambda x{:}T.t \mid t \, t \\
					  %&   &     & \If{t}{t}{t} \\
		\text{Types:} & T & ::= & ? \mid T \rightarrow T \\
		\\			 
		\text{Typing:} & \frac{x:T \in \Gamma}{\Gamma \vdash x : T}
					   & \frac{\Gamma,x:T_1 \vdash e : T_2}
					          {\Gamma \vdash \lambda x:T_1.e : T_1 \to T_2} 
					   & \frac{\Gamma \vdash t_1 : T_1 \rightarrow T_3 \quad 
					           \Gamma \vdash t_2 : T_2 \quad
					           T_2 \sim T_3}
					          {\Gamma \vdash t_1 \, t_2 : T_3}\\\\
					  & & & \frac{\Gamma \vdash t_1 : ?}
					          {\Gamma \vdash t_1 \, t_2 : ?}\\
		\\
	    \text{Consistency:} & \frac{T_1 = T_2}{T_1 \sim T_2}
	      					& T \sim ? & ? \sim T\\
		\\
		&& \frac{T_1 \sim T_3 \quad T_2 \sim T_4}
		        {T_1 \rightarrow T_2 \, \sim \, T_3 \rightarrow T_4}
	\end{array}
	\]
	\caption{Static Type Checking rules for the GTLC}
	\end{figure}
}

\frame{
  \frametitle{Compilation from the GTLC to the Cast-Caculus}
  \begin{figure}[h]
	\centering
	\begin{gather*}
	   \frac{x : T \in \Gamma}
	       	{\Gamma \vdash x \rightsquigarrow x : T} 
	   \quad
	   \frac{\Gamma x : T_1 \vdash t \rightsquigarrow t^{\prime} : T_2}
	        {\Gamma \vdash \lambda x {:} T_1 . t \rightsquigarrow 
	                       \lambda x {:} T_1 . t^{\prime} : T_1 \rightarrow T_2}\\\\
	   \frac{\Gamma \vdash t_1 \rightsquigarrow 
	                       t_1^{\prime} : T_1 \rightarrow T_3 \quad 
			 \Gamma \vdash t_2 \rightsquigarrow t_{2}^{\prime} : T_2 \quad
			  T_1 \sim T_2}
			{\Gamma \vdash t_1 \, t_2 \rightsquigarrow
			               t_1^{\prime} \, (t_2^{\prime} : T_2 \Rightarrow 
			               	                               T_1) : T_3}\\\\
	   \frac{\Gamma \vdash t_1 \rightsquigarrow 
	                       t_1^{\prime} : ? \quad 
			 \Gamma \vdash t_2 \rightsquigarrow t_{2}^{\prime} : T_2 \quad}
			{\Gamma \vdash t_1 \, t_2 \rightsquigarrow
			               (t_1^{\prime} : ? \Rightarrow T_2) t_2^{\prime} : ?}\\\\
	\end{gather*}
	\caption{Rules for Compilation}
	\end{figure}
}

\frame{
  \frametitle{Introducing an Implementation of Casts}
  \begin{figure}[h]
	\centering
	\begin{gather*}
	   (v \, T \Rightarrow T) \longrightarrow v
	   \\\\
	   (u \, T \Rightarrow ?) \longrightarrow (u,\, T)
	   \\\\
	   ((u, \, T_1) \, ? \Rightarrow T_2) \longrightarrow 
	      (u \, T_1 \Rightarrow T_2) 
	   \\\\
	   (v \, T_1 \rightarrow T_2 \Rightarrow T_3 \rightarrow T_4) \longrightarrow
	      \lambda x:T3. ((v \, (x \, T_3 \Rightarrow T_4) \, T_2 \Rightarrow T_4)
	   \\\\
	   \frac{ (T_1 {\nsim} T_2)}
	        {(v \, T_1 \Rightarrow T_2) \longrightarrow \, error}
	\end{gather*}
	\caption{Rules for Compilation}
	\end{figure}
}

\frame{
	\frametitle{Why Are There Other Implementations of Casts}
}

\frame
{
  \frametitle{Goals for the behavior of GTLC?}
  \begin{itemize}
  \item A GTLC program with no occurrences of the dynamic type, aka fully typed, 			should behave the same as a program of the Simply-Typed Lambda Calculus
  \item Erasing types by adding the dynamic type should not change the observable 
        behavior of a fully typed program that type checks.
  \item Loftier Goals ...
 	\begin{itemize}
	\item Static Programs should be fast (What is fast?)
	\item Programs have simple performance models
	\end{itemize}
  \end{itemize}
}


\frame{
	\frametitle{The Schml Compiler}
	\begin{itemize}
	\item From S-Expression version of the Gradually-Typed Lambda Calcucus
	\item Written in Typed Racket
	\item Targeting C
	\end{itemize}
}

\frame{
	\frametitle{Goals for the Schml Compiler}
	\begin{itemize}
	\item Clear and Concise Implementation of Various Semantics
	\item Measure Performance
	\item Find Optimizations 
	\end{itemize}
}

\frame{
	\frametitle{State of the Schml Compiler}
	\begin{itemize}
	\item Not a Compiler Yet (Fancy Interpreter)
	\item Implementation of One of the Semantics
	\item Doesn't Yet Measure Performance
	\item Found a Few Optimizations
	\end{itemize}
}

\frame{
	\frametitle{Thank You}
}

\end{document}